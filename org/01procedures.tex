% Created 2023-12-17 Sun 17:27
% Intended LaTeX compiler: pdflatex
\documentclass[11pt]{article}
\usepackage[utf8]{inputenc}
\usepackage[T1]{fontenc}
\usepackage{graphicx}
\usepackage{longtable}
\usepackage{wrapfig}
\usepackage{rotating}
\usepackage[normalem]{ulem}
\usepackage{amsmath}
\usepackage{amssymb}
\usepackage{capt-of}
\usepackage{hyperref}
\author{Caesar X Insanium}
\date{\today}
\title{Chapter 1: Building Abstractions with Procedures}
\hypersetup{
 pdfauthor={Caesar X Insanium},
 pdftitle={Chapter 1: Building Abstractions with Procedures},
 pdfkeywords={},
 pdfsubject={},
 pdfcreator={Emacs 29.1 (Org mode 9.6.6)}, 
 pdflang={English}}
\begin{document}

\maketitle
\tableofcontents

Here are the relevant lectures related to this chapter

\begin{itemize}
\item \href{https://www.youtube.com/watch?v=eJeMOEiHv8c}{Lecture 1A}
\end{itemize}
\href{https://www.youtube.com/watch?v=-J\_xL4IGhJA\&list=PLE18841CABEA24090\&index=1}{- Lecture 1B}
\begin{itemize}
\item \href{https://youtu.be/eJeMOEiHv8c?list=PLE18841CABEA24090}{Lecture 2A}
\end{itemize}

A process is an idea of a set of events occurring under the direction of
a program.

Lips was invited as a way to express computational thought in a manner
using recursion equations. Lisp is an interpreted language that is akin
to a mathematical notation. While it used to be slow it became able to
translate directly to machine code in order to get that boost in
performance. It can now be used in production and as an extension
language. In lisp, data and function can be both be returned from
function calls.

\section{1.1 Elements of Programming}
\label{sec:org5c8b2d9}
Programming languages can be used to express ideas about instructions
that we want a computer to execute as well as about processes and
incorporate decision-making. This is composed to 3 things

\begin{itemize}
\item primitive expressions: simplest entities and data
\item combination: methods to build larger things from smaller ones
\item abstractions: ways to reduce complexity by finding implementation
details and organizing data and procedures
\end{itemize}

\subsection{1.1.1 Expressions}
\label{sec:orge950840}
Expressions are evaluated and interpreter will return a value, usually
itself.

Here is an example in lisp

\begin{verbatim}
89 ;; will return 89 as an integers
(+ 9 10) ;; will evaluate result of adding 9 and 10 and then return
(+ 1 1 1 1) ;; unlimited number of operands
(+ (* 3 8) (- 10 4)) ;; nested expressions
\end{verbatim}

Operators and operands are organized into prefix notation where the
operator is listed first and operands come afterwards. This hold true
for function calls. More complicated expressions are difficult for
humans to understand, but an interpreter can easily handle complexity.

Of course such expressions can also be rewritten in a way that would
allow for easier human understanding by aligning operators.

\begin{verbatim}
(+ (* 3
      (+ (* 2 4)
         (+ 3 5)))
   (+ (- 10 7)
      (6))
\end{verbatim}

\subsection{1.1.2 Naming and Environment}
\label{sec:orgc5b7642}
In lisp things can be named with \texttt{define}. This allows easy reference
with a value without spelling it out. Assigned variable names are
evaluated with their referred value.

\begin{verbatim}
(define size 2)
(+ size 2) ;; evaluates to 2
\end{verbatim}

\subsection{1.1.3 Evaluating Combinations}
\label{sec:orgd1d791c}
In order to evaluate lisp expression

\begin{itemize}
\item evaluate sub-expressions
\item apply procedure on data
\item return value
\end{itemize}

Imagine a tree structure that with each expression, there is an operator
and a list of values. They can either be literals or expressions that
themselves need to be evaluated.

\subsection{1.1.4 Compound Procedures}
\label{sec:org923e289}
Abstraction technique that allows for a set of instructions to be saved
and used multiple times in a program.

Here is procedure that square a number.

\begin{verbatim}
(define (square x) (* x x))
;; here is example usage
(square 21)
(square (+ 6 7))
(square (square 5))
\end{verbatim}

Define a procedure where we multiply a thing by itself. Here we use the
\texttt{define} keyword to name a procedure, specify that it takes only one
parameter and then following expression is that happens to data that is
passed.

\begin{verbatim}
(define (<name> <formal parameters>) (<body expression>))
\end{verbatim}

\subsection{1.1.5 Substitution Model Procedure Application}
\label{sec:orgd3b2c6d}
Evaluation begins from the top down, in order to begin to evaluate the
top most expression, first the bottom expressions must be evaluated
recursively. One way is to replace every instance where we call a
procedure with the body of said procedure.

One can simply think back to algebra to when we were solving systems of
equations. This is known as the substitution model.

Full implementation of interpreter and compiler will be done in this
book.

Normal order evaluation is done by humans where full meaning of
expression is expanded to using only the most basic ideas and then
evaluate. Interpreter use application order evaluation where expressions
are evaluated before applying.

\subsection{1.1.6 Conditional Expression and Predicates}
\label{sec:org3825554}
Method for executing different based on results of tests if required for
Turing completeness and is basis for any for decision-making. Here are
some examples.

\begin{verbatim}
(define (abs x)
  (cond ((> x 0) x) ;; this function is very similar to switch statement from C
        ((= x 0) 0)
        ((< x 0) (-x))))
(define (abs x)
  (cond ((< x 0) (-x))
        (else x)))
(define (abs x)
  (if (< x 0)
      (- x)
      x))
\end{verbatim}

Here a condition is defined followed by a list of clauses or tests that
can be performed.

We see the base form for a conditional statement in lisp.

\begin{verbatim}
(if <predicate> <consequent> <alternative>)
\end{verbatim}

\subsubsection{Example: Square Root using Newton's Method}
\label{sec:orgb19a491}
\begin{verbatim}
(define (square x)
    (* x x))

(define (improve guess x)
  (average guess (/ x guess)))

(define (average x y)
  (/ (+ x y) 2))

(define (good-enough? guess x)
    (< (abs (- (square guess) x)) 0.001))

(define (sqrt-iter guess x)
  (if (good-enough? guess x)
    guess
    (sqrt-iter (improve guess x)
                x)))

(define (sqrt x)
  (sqrt-iter 1.0 x))
\end{verbatim}

\subsection{1.1.8 Procedures as Black-Box Abstractions}
\label{sec:orge028c92}
Procedure Abstraction should allow for encapsulation of lower level
procedure and joining them together in order to make larger procedures.
Naming arguments generally does not matter to user of a procedure. All
that matters is that the function is correct and returns needed value.

Lisp also allows defining private procedures inside other procedures in
order to be able to maintain a specific procedure that is important to
working of public facing function.

\begin{verbatim}
(define (public x)
  (define (private y)
    (write y)
    (newline))
  (private x)
\end{verbatim}

\section{1.2 Procedures and Processes They Generate}
\label{sec:orgb0aad1b}
In order to make the best use of procedure it is not enough to know how
they work one must also know tactics and strategies for using them. In
order to get the bast idea on how to design and build a system one must
have a good idea on what we want the end result to be. It makes it
necessary to plan out much of what we want to do. Procedures are local
evolutions of computational processes and as such can be built on top of
each other in order to create the bigger result. They are some "shapes"
that procedure definitions can follow.

Lisp allows reasoning about and build procedures as if they were
mathematical expressions.

\subsection{1.2.1 Linear Recursion}
\label{sec:orge403bc2}
The definition of factorial is as follows

\begin{equation}
n! = n * (n - 1) * (n -2) ...3 * 2 * 1
\end{equation}

From this it is logical to assume that \emph{n} factorial is equal to \emph{n}
times \emph{n} minus one factorial.

\begin{equation}
n! = n * (n-1)!
\end{equation}

Here is a more recursive method

\begin{verbatim}
(define (factorial n)
  (if (= n 1)
    1
    (* n (factorial (- n -1)))
  )
)
\end{verbatim}

Here is another method of defining the factorial function.

\begin{verbatim}
(define (factorial n)
  (define (iter product counter)
    (if (> counter n)
      product
      (iter (* counter product)
            (+ counter 1))))
  (iter 1 1))
\end{verbatim}

Expanding expression allows one to see the true "shape" of a procedure
as it evaluates. Some have a diamond shape as the expression expands to
the simplest term and then contracts as each term is evaluated.

The expansion evaluation of a procedure is known as
\texttt{deferred operations} in which an is like as the program is being run
this is the part where each function is being initial called. And then
after that the evaluation of the value is the actual recursion.

With the iterative function definition each time the recursive function
is called it is being immediate evaluated before it can be called again.
This is \texttt{iterative} processes. In this method the state can be tracked
with certain variables keeping track of when the process should end if
at all. For is a for loop in most languages.

Iterative procedures allow for easy description of the entire state of
the process at any given point. Recursive not so much since the
compiler/interpreter hide away much of the inner working up to the point
in which it is difficult to resume from any position.

Recursive procedures are strictly those that call themselves within
their own definition. Iterative procedures hold their state in outside
variables.

Most languages define iterative procedures using for/while loop unlike
Scheme and any other lisp dialect.

\subsection{1.2.2 Tree Recursion}
\label{sec:org854cb8f}
\emph{Tree recursion} occurs when a procedure calls itself more than once
inside its own definition. For example look at this procedure for
getting Fibonacci numbers.

\begin{verbatim}
(define (fib n)
  (cond ((= n 0) 0)
        ((= n 1) 1)
        (else (+ (fib (- n 1))
                 (fib (- n 2))))))
\end{verbatim}

This example is not efficient because work is repeated. Computing the
Fibonacci of any sufficiently big number will result with entire
branches of work being recalculate in different branches in the
execution of the procedure. The time complexity of this function is
exponential. Space complexity grow linearly.

Here is iterative version.

\begin{verbatim}
(define (fib_i n)
  (fib-iter 1 0 n))

(define (fib-iter a b counter)
  (if (= counter 0)
    b
    (fib-iter (+ a b) a (- ounter 1))))
\end{verbatim}

Here the time complexity is linear.

Tree recursion and iteration are tools and should be used when it is the
best tool in the current situation.

\subsection{1.2.3 Orders of Growth}
\label{sec:orgcd36e3d}
Order of growth simply describe who quickly the time it takes for a
procedure to finish executing given data size \emph{n} as \emph{n} reaches
infinity. \emph{N} can be used to describe the size of a number itself, the
number of bit required to describe a piece of data, the number of
elements in a data structure.

Such method of describing a procedure grows in computation time is
inaccurate, but it is useful in describing how efficient an algorithm
really is.

\subsection{1.2.4 Exponentiation}
\label{sec:orga58f381}
One way to calculate the exponent of a value is to use the recursive
definition of exponentiation.

\begin{equation}
b^n = b * b^{n-1},
b^0 = 1
\end{equation}

Here it is in scheme.

\begin{verbatim}
(define (expt b n)
  (if (= n 0)
    1
    (* b ( expt b (- n 1)))))
\end{verbatim}

It is possible to make a faster procedure by simply using a different
algorithm. It can be done using the idea that certain value can be
reached faster. For example take following expression.

\begin{equation}
b^4 = b^2 * b^2
\end{equation}

Using this \[ b^8 \] can be calculated much faster. Here is an
implementation.

\begin{verbatim}
(define (fast-expt b n)
  (cond ((= n 0) 1)
        ((even? n) (square (fast-expt b (/ n 2))))
        (else (* b (fast-expt b (- n 1))))))
\end{verbatim}

Given from this implementation, it can be easy to see that the time
complexity goes to logarithmic because each exponent jump is bigger the
deeper the recursion goes.

\subsection{1.2.5 Greatest Common Divisor}
\label{sec:org0fe42aa}
Greatest Common Divisor or GCD is defined as the largest integers that
divides integers A and B. Meaning that there are integers x and y such
that \[ x *GCD =A \] and \[ y* GCD =  B \]

Finding the GCD is simple since one can take the recursive definition.



\begin{equation}
GCD(a,b) = GCD(b,r)
\end{equation}

Use it to define this function in scheme.

\begin{verbatim}
(define (gdc a b) (if (= b 0)
                    a
                    (gcd (remainder a b))))
\end{verbatim}

We continuously divide a from b recursively until the final GCD is
found.

This algorithm is iterative in nature as each recursion is a tail
recursion and grows in logarithmic time.

\subsubsection{Example: Testing for Primality}
\label{sec:orgce9b46d}
There are two main methods for testing if integer n is a prime number or
not.

One strategy is to find if it has any factors.

\begin{verbatim}
;;Naive Method

(define (square x) (* x x))

(define (smallest-divisor n)
  (find-divisor n 2))

(define (divides? a b) (= (remainder b a) 0))

(define (find-divisor n test-divisor)
  (cond ((> (square test-divisor) n ) n)
        ((divides? test-divisor n) test-divisor)
        (else (find-divisor n (+ test-divisor 1)))))

(define (prime? n) (= n (smallest-divisor n)))
\end{verbatim}

What this implementation does it check all numbers less than square root
of n if they are the prime numbers. The logic goes if none of those work
then the number \emph{n} is prime.

The second big strategy is based on Fermat's Little theorem which states
basically the idea that is \emph{n} is a prime number and \emph{a} is a positive
integer less than \emph{n} then \emph{a} raised to the \emph{n} power is congruent to
\emph{a} modulo \emph{n}.

Two numbers are congruent if when divide by the same number have the
same remainder.

\begin{verbatim}
;; Fermats Theorem applied

(define (expmod base exp m)
  (cond ((= exp 0)
         1)
        ((even? exp)
         (remainder (square (expmod base (/ exp 2) m)) m))
        (else
          (remainder (* base (expmod base (- exp 1)m))m))))
\end{verbatim}

This procedure grows logarithmic in accordance to size of input. Using
it we can implement the Fermat test.

\begin{verbatim}
(define (fermat-test n)
  (define (try-it a)
    (= (expmod a n n) a))
  (try-it (+ 1 (random (- n 1)))))
\end{verbatim}

The final form. Will take a number and number of attempts to see if it
is prime.

\begin{verbatim}
(define (fast-prime? n times)
  (cond ((= times 0) 1)
        ((fermat-test n) (fast-prime? n (- times 1))) (else 0)
\end{verbatim}

However, it has been proven that even if the procedure says that a
number passes the test then it does not mean that it is prime. There is
also the fact that some non primes that fool the Fermat test for why
modulo congruent with a lot of numbers. This algorithm is an example of
a \texttt{Probabilistic Algorithm} in which there is a chance of error that the
algorithm yields the incorrect result.

\subsection{Formulating Abstractions with Higher Order Procedures 1.3}
\label{sec:orgec1fa23}
The ability to write procedures and function allows for the ability to
create program that can work on higher and higher levels of abstraction
and reuse instructions and operations without repeat the definitions.

Procedures that only work on numbers can be limiting. \texttt{Higher Order}
procedures are those that accept functions as arguments and can return
functions as results.

\subsection{1.3.1 Procedures as Argument}
\label{sec:orgd5a7624}
Here we have three procedures.

\begin{verbatim}
;; Sums Integers from A to B
(define (sum-integers a b)
        (if (> a b)
            0
            (+ a (sum-integers (+ a 1) b))))

;; Computers Sum of Cubes of Integers
(define (sum-cubes a b)
    (if (> a b)
        0
        (+ (cube a) (sum-cubes (+ a 1) b))))

;; Computes Series of multiplied cubes
;;; 1 / (1 * 3) + 1 / (5 * 7) +1 / (9 * 11)
(define (pi-sum a b)
    (if (> a b)
        0
        (+ (/ 1.0 (* a (+ a 2))) (pi-sum (+ a 4) b))))
\end{verbatim}

These are similar procedures, that have the same template and could be
automatically generated. Sigma notation is used in order to express how
express a method to add the results of a function given a range of
integer values.

\begin{equation}
\sum_{n=a}^{b}(f(n) = f(a)+...+f(b))
\end{equation}

The realization that the pattern being emulated is in fact a
mathematical summation allows for easy redefinition using the scheme
language.

\begin{verbatim}
(define (sum term a next b)
        (if (> a b)
            0
            (+ (term a)
                (sum term (next a) next b))))

;; term is a function that determines the selection of items being summed
;; next is a function that determines which is the next items after the previus one
    ;; for function term
\end{verbatim}

Using some other helper functions it is possible to redefine all the
functions.

\begin{verbatim}
;; redefinition of a previous function
(define (sum-cubex a b)
        (sum cube a inc b))

(define (sum-integers a b)
        (sum identity a inc b))

(define (pi-sum a b)
        (define (pi-term x)
                (/ 1.0 (* x (+ x 2))))
        (define (pi-next x)
                (+ x 4))
        (sum pi-term a pi-next b))
\end{verbatim}

Using this same higher order procecure it is possible to define a
function to approximate the integral of a function.

\begin{verbatim}
(define (integral f a b dx)
        (define (add-dx x)
                (+ x dx))
        (* (sum f (+ a (/ dx 2.0)) add-dx b)
            dx))
;; can be used to approximate the integral from 0 to 1 of x^3
(integral cube 0 1 0.01)
;; .24998750000000042 result
(integral cube 0 1 0.001)
;; .249999875000001 result
\end{verbatim}

\subsection{1.3.2 Constructing Procedures Using Lambda}
\label{sec:orge0b7f2d}
Scheme allows for methods of defining simple single use functions
without giving them names. These are \texttt{lambda}, anonymous function that
are a definition of a function that does one simple thing and developer
forgets about them.

\begin{verbatim}
(lambda (x) (+ x 4))
;; Format
(lambda (<name> <formal-parameters>) <body>)
\end{verbatim}

Usually these are passed to other functions in order to generate higher
level procedures and are the hallmark of a functional programming
language. An example using previous mentions.

\begin{verbatim}
(define (pi-sum a b)
        (sum (lambda (x) (/ 1.0 (* x (+ x 2))))
            a
            (lambda (x) (+ x 4))
            b))

(define (integral f a b dx)
        (* (sum f
                (+ a (/ dx 2.0))
                (lambda (x) (+ x dx))
                b)
            dx))
\end{verbatim}

The usualy method for defining named function can be thought of as
syntactic sugar for lambda.

\begin{verbatim}
(define (f x) (+ x 1)) ;; is the same as
(define f (lambda (x) (+ x 1)))
\end{verbatim}

\subsubsection{Using \texttt{let} to create local variables}
\label{sec:orgda032b8}
Using the \texttt{let} keyword is useful in defining variables with limited
scope in order to pollute namespace. Taking the mathematical expression

\begin{equation}
f(x, y) = x(1 + (x * y))^2 + 1 y (1 - y) + ( 1 + (x * y))(1 - y)
\end{equation}

can be simplified to

\begin{equation}
a = 1 + (x * y),
b = 1 - y,
f(x, y) = xa^2 + (y * b) + ab
\end{equation}

Writing scheme code in order to emulate this function would require not
only the parameters but also defining the local variables a and b. Here
is some normal scheme code.

\begin{verbatim}
(define (f x y)
        (define (f-helper a b)
                (+ (* x (square a))
                   (* y b)
                   (* a b)))
        (f-helper (+ 1(* x y))
                  (- 1 y)))
\end{verbatim}

Here is the simplification. Side note, getting the indentation correct
was a pain in the ass.

\begin{verbatim}
(define (f x y)
    (let ((a (+ 1 (* x y)))
          (b (- 1 y)))

         (+ (* x (square a))
            (* y b)
            (* a b))))
\end{verbatim}

General form for let expression is this

\begin{verbatim}
(let ((<var1> <exp1>)
      (<var2> <exp2>)
      ...
      (<varn> <expn>))
    <body>)
\end{verbatim}

Each section can be thought of as its own little mini section with
define pairs. The different is that the variables here have limited
scope and exist only within the confines of the S expression of let.
Even this expression can be thought of as syntactic sugar for a lambda
expression that take in certain parameters that are the names of
parameters and outputs a list with each item being corresponding to the
value of expression.

If a variable already has a value, then the \texttt{let} expression overrides
it within the scope of the expression, ignoring outside values.

\begin{verbatim}
(define x 5)
(+ (let ((x 3))
        (+ x (* x 10)))
   x) ;; evaluates to 38
\end{verbatim}

Best practices dictates to use \texttt{let} in defining variables and \texttt{define}
in defining internal procedures.

\subsection{1.3.3 Procedures as General Methods}
\label{sec:orgf1e6e5c}
Strategies have been introduced that allow for expression numerical
methods as scheme procedures, and abstracting away general strategies to
be independent of numbers and or even computation involved.

\subsubsection{Finding Roots of Equations by Half Interval Methods}
\label{sec:org626f73e}
Half interval method involved in finding roots of function which average
in range of values is found and if prove that a zero exists in interval,
the interval is cut in half until the root is found.

The number of steps required to find the root grows logarithmic in
relation to size of starting interval. Here is example from book.

\begin{verbatim}
(define (search f neg-point pos-point)
  (let ((midpoint (average neg-point pos-point)))

       (if (close-enough? neg-point pos-point)
           midpoint
           (let ((test-value (f midpoint)))
                (cond ((positive? test-value)
                       (search f neg-point mid-point))
                      ((negative? test-value)
                       (search f midpoint pos-point))
                      (else midpoint))))))
\end{verbatim}

What this code does is first calculated the midpoint of the interval.
Then it checks to see if the interval is close enough, if it is, it
returns the midpoint. If not, then it evaluates the function f at the
midpoint. If this new value is positive then it calls itself with new
interval from beginning to midpoint. Then if negative the new range is
from midpoint to end of range. If the midpoint evaluates to zero then it
gets returned.

Search should not be called directly and instead suitable interval
should first be found and then \texttt{search} should be called.

\subsubsection{Fixed Points of Functions}
\label{sec:org9499a2b}
Fixed point in function are where the output of a function is the same
as the input. These points can be found by applying this calculation.
Where x is a guess and applying the same transformation.

\begin{equation}
f(x), f(f(x)), f(f(f(x))), f(f(f(f(x))))
\end{equation}

Using this definition it is possible to define procedure that finds a
fixed point given a function and initial guess. Here is the example the
book gives.

\begin{verbatim}
(define tolerance 0.00001)

(define (close-enough? x y)
  (< (abs (- x y)) tolerance))
(define (fixed-point f first-guess)
  (define (try guess)
    (let ((next (f guess)))
         (if (close-enough? guess next)
             next
             (try next))))
  (try first-guess))

(fixed-point cos 1.0)
\end{verbatim}

This code basically just guesses the fixed point until it gets it close
enough based on some criteria.

It can also find the solution to equations. Give for example.

\begin{equation}
y = cos(y) + sin(y)
\end{equation}

Can be solved with.

\begin{verbatim}
(fixed-point (lambda (x) (+ (sin x) (cos x))) 1.0)
\end{verbatim}

Average Damping can also be used in order to more carefully converged
towards the wanted value without getting stuck in an infinite loop.

\subsection{1.3.4 Procedures as Returned Values}
\label{sec:org58486ce}
More power and expression ability can be achieved by using procedures
that generate and return procedures. For example, procedure can be
defined that can average damp the result of a function.

\begin{verbatim}
(define (average-damp f)
  (lambda (x) (average x (f x))))
\end{verbatim}

What this function does is it take a function and return a new function
that takes the average of an input and the result of calling that
function on an input. This concept can be used in order to redefine the
square root function in terms of the previously define \texttt{fixed-point}
function.

\begin{verbatim}
(define (sqrt x)
  (fixed-point (average-damp (lambda (y) (/ x y))) 1.0))

;; Cube Root can also be defined
\end{verbatim}

Recognizing patterns that repeat in procedures is an important skill to
have as it for more easy reuse of different components.

\subsubsection{Newton's Method}
\label{sec:org3343891}
Early implementation of square root function involved Newton's method in
which solution can be approximated using the identity.

\begin{equation}
f(x) = x - \frac{g(x)}{Dg(x)}
\end{equation}

The does here is to find \emph{x} such that f(x) = 0. The bottom term is the
derivative of the function \emph{g} at \emph{x}. This method does not always
converge but does converge enough times such that it is useful
approximation.

In order to use this method, the derivative must found, and used. It is
defined as

\begin{equation}
g'(x) = \frac{g(x + dx) - g(x)}{dx}
\end{equation}

The code here is simply returning a new function that is in terms of the
passed and uses the definition of the derivative in order to calculate
the derivative at a certain value of \emph{x}.

\begin{verbatim}
(define dx 0.000001)
(define (deriv g)
  (lambda (x) 
        (/ (- (g (+ x dx))
              (g x))
           dx)))
\end{verbatim}

A new procedure can be defined that take a function \emph{f} and returns it's
derivative in terms of \emph{f}. Then Newton's Method can be implemented by
using it as a fixed point function.

\begin{verbatim}
(define (newton-transform g)
  (lambda (x)
          (- (/ (g x)
                ((derive g) x)))))

(define (newtons-method g guess)
  (fixed-point (newton-transform g) guess))
\end{verbatim}

The square root function can be redefined in terms of these new
functions.

\subsubsection{Abstractions and First Class Procedures}
\label{sec:org9f58054}
Further abstraction can be achieved by formulating the repeat used of
the fixed point function and abstracting its repeated use.

\begin{verbatim}
(define (fixed-point-of-transform g transform guess)
  (fixed-point (transform g) guess))
\end{verbatim}

This procedure takes as arguments function g, a function that a solution
found, and an initial guess. What is does is it calls the fixed point
function that takes the transform procedure and calls it on g, and the
guess. The whole purpose is to define a function that can be solved and
to solve it. And this can be used to redefine the square root function.

\begin{verbatim}
(define (sqrt x)
  (fixed-point-of-transform (lambda (y) (/ x y))
                            average-damp
                            1.0))
\end{verbatim}

These higher order procedures allow for easy expression of computing
methods using more basic elements that already been defined in our code.
Overly abstraction code can also de detrimental but the price of not
doing so is worse. The key is thinking in terms of abstraction and
reusing them whenever code begins to feel repetitive and to apply them
and new contexts.

In computer languages there is a concept of first-class elements. These
have properties in computer languages in which there are few
restrictions and have some features.

\begin{itemize}
\item Can be named as variables
\item passed as arguments to functions
\item returned as result of procedures
\item include in data structure
\end{itemize}

In lisp and other dialects function are first class citizens, with all
the above features. Fast and efficient implementation is difficult, but
expression are very powerful.
\end{document}